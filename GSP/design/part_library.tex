\chapter{연습 준비}

연습을 위한 준비와 연습을 위한 방법을 정리한다. 

\begin{itemize}
    \item ide 준비하기 
    \item vcpkg
    \item doctest 사용법
    \item 디버깅 
    \item 코드 읽기 
\end{itemize}

ide는 visual studo와 visual c++로 한다. 일반적으로 쉽게 접근 가능하므로 이를 기준으로 
하고 vscode + cmake + clang은 가볍게 언급한다. 이 쪽으로 진행하는 사람도 있을 수 있다. 

doctest는 단위 테스트로 기능을 꼼꼼하게 살펴서 연습하는 방법으로 설명한다. 
단위 테스트와 assert에 해당하는 CHECK로 연습하게 한다. TDD를 가볍게 언급하고 현실적으로 
적용 가능한 수준에 대해 정리한다. 

\chapter{라이브러리 목록}

게임 서버 개발에 필수적인 라이브러리들로 이미 사용했거나 사용할 것들을 
정리하고 연습으로 만든다. 

\section{표준 라이브러리}

\begin{itemize}
    \item container 
    \item string 
    \item memory 
    \item file 
    \item thread 
    \item lock
    \item atomic
    \item chrono 
    \item regex 
\end{itemize}

\section{boost와 신규 라이브러리}
std에 없는 것들을 언급할 필요가 있다. 찾아서 연습할 수 있도록 한다. 

\begin{itemize}
    \item boost multi\_index 
    \item redis skiplist
\end{itemize}

\section{tbb}

이외에 tbb에 있는 컨테이너들도 중요하고 유용하다. 

\begin{itemize}
    \item concurrent\_queue, concurrent\_map 
\end{itemize}


\section{메모리 관리}

메모리 관리 라이브러리들도 중요하다. 

\begin{itemize}
    \item tcmalloc 
    \item mimalloc
\end{itemize}

\section{통신}

통신과 관련한 기본 라이브러리들이다. 

\begin{itemize}
    \item asio 
    \item cpr 
    \item sodium 
    \item crypto++
    \item flatbuferrs
\end{itemize}

\section{데이터 관리}

데이터 관리를 위한 라이브러리이다. 

\begin{itemize}
    \item RapidCsv
    \item nlohmann json 
    \item nanodbc
    \item redis 
\end{itemize}


\section{공간과 물리 처리}

공간과 물리 처리를 위한 라이브러리이다. 

\begin{itemize}
    \item recastdetour 
    \item bulletphysics
    \item box2d
\end{itemize}


\section{메트릭}

메트릭 라이브러리들이다. 

\begin{itemize}
    \item prometheus 
    \item grafana 
    \item exporter 들
\end{itemize}

\section{툴 개발}

툴 개발을 위해 필요한 라이브러리이다. 

\begin{itemize}
    \item imgui
\end{itemize}