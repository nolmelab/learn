\chapter{무엇을 쓸 것인가?}

\section{안내를 한다}

게임 서버 프로그래밍에 필요한 내용들을 찾아서 이해하고 사용할 수 있는 연습을 
할 수 있도록 안내를 한다. 

연습은 반복을 한다는 뜻을 함께 갖는다. 이해도 반복해서 하면 더 깊어지고, 사용도 
반복해서 하면 더 능숙하게 된다. 

\section{정리를 한다}

안내를 위해서는 잘 알아야 하므로 찾고 관계를 이해하고 내용을 정리해서 
안내를 준비한다. 

두 과정이 잘 맞물리면 점점 좋은 내용으로 발전할 수 있다. 

\section{실제 내용들}

게임 개발을 위해 알아야 하는 내용들은 많다. 따라서, 모든 내용을 정리하고 연습으로 
만들기는 어렵다. 그러므로 스스로 찾아 갈 수 있는 훈련이 되도록 하고 단단한 기초가 
되는 부분들은 확실하게 정리하면 좋다. 

실제 개발을 하면서 필요로 했던 내용들과 필요하지만 아직 명확하게 정리되지 않은 것들을 
중심으로 정리하고 쓴다. 

\begin{itemize}
    \item 라이브러리 
    \item 통신
    \item 데이터베이스 
    \item 분산 처리 
    \item 게임 데이터 
    \item 게임 시스템 
\end{itemize}

