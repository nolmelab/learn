\chapter{어떤 라이브러리가 필요할까요?}

게임 서버를 만들 때 필요한 라이브러리는 다양하게 많습니다. 많은 사람들이 모여서 
함께 살아가는 새로운 가상의 세계를 만들려면 통신, 데이터베이스가 필요합니다. 또 
안정적으로 동작하는지 확인하려면 로그를 잘 남기고 서버 상태를 알 수 있어야 합니다. 

그래서 필요한 도구들을 현재까지 사용해 온 경험을 살펴서 정리하면 다음과 같습니다. 

\begin{itemize}
    \item doctest 단위 테스트  
    \item C++ 표준 라이브러리 (Container, Thread 등)
    \item fmt 또는 format 
    \item spdlog 로깅  
    \item asio 통신과 처리 (Proactor)
    \item nanodbc 데이터베이스 ODBC 라이브러리 
    \item boost에서 필요한 라이브러리들
\end{itemize}

여기까지가 기본 중의 기본이 되는 라이브러리입니다. asio로 서버 처리와 TCP 통신을 
작성할 수 있고 DB에 저장하고 읽을 수 있다면 기본은 된 상태라 할 수 있습니다. 

이 상태에서는 가상의 물리적인 월드 구성이 없는 모든 게임을 만들 수 있습니다. 예를 들어, 
멀티플레이어 테트리스, 마비노기 영웅전 (물리적인 세상이 있지만 클라이언트에서 처리), 
카트 라이더 (물리적인 세상이 있지만 클라이언트에서 처리) 등이 있습니다. 

그 위에 세상을 만들기위해 이동과 길찾기, 충돌 처리 라이브러리를 필요로 합니다. 
주로 쓰이는 라이브러리는 다음과 같습니다. 

\begin{itemize}
    \item RecastDetour 이동과 길찾기 
    \item bulletphysics 충돌과 물리
\end{itemize}

RPG (Role Playing Game) 장르의 게임들은 서버에서 대부분 위 라이브러리나 유사한 기능을 
서버에서 구현해야 합니다. 게임마다 선택은 다르지만 위 두 가지를 기초로 합니다. 

이동과 길찾기, 충돌과 물리 처리는 유니티나 언리얼과 같은 클라이언트 엔진에 잘 구현되어 
있습니다. 게임 서버에서는 더 빠르게 처리해야 많은 사용자를 처리할 수 있으므로 여러 방법을 
동원하여 가볍게 만들려는 노력을 합니다. 





