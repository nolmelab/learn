\chapter{연습을 준비하기}

\section{이해하고 연습하는 과정}

게임 서버를 만들기위해 필요한 라이브러리와 도구는 항상 변화하고 발전합니다. 한 때는 
boost를 사용하는 걸 꺼리던 적도 있고, 지금은 표준 라이브러리에 있는 std::map과 같은 
일반화된 라이브러리 대신 직접 만들어 쓰던 적도 있습니다. 

아직 언리얼 엔진을 보면 직접 컨테이너를 만들거나 공유 포인터를 직접 작업하고 쓰고 
있습니다. 꼭 필요하다면 할 수도 있는 선택이지만 게임 서버는 대체로 표준화되고 
안정적인 라이브러리를 사용합니다. 

대신 게임의 차별성을 위해 꼭 필요한 기능들을 새롭게 만들어야 하고, 그런 기능에 
집중하는 것이 대체로 낫습니다. 

\subsection{단위 테스트로 이해하기}

대신 내부적인 구현이 어떻게 되어있는지, 성능 특성은 어떤지, 잘못 쓰기 쉬운 부분은 
어떤지 등 잘 이해하고 써야 합니다. 그러기 위해서 이미 알려진 내용들을 확인하고 
직접 단위 테스트 안에서 써보면서 이해합니다. 

또 단위 테스트에서 디버깅을 하면서 실행 시간에 코드를 따라가는 연습을 하면서 
코드에 대한 이해를 더 자세하게 할 수 있도록 합니다. 

\subsection{C++ 표준 문서 읽기}

C++ 표준 문서는 형식이 매우 체계화되어 있고 내용이 방대하여 읽기가 어렵습니다. 
하지만, 표준을 읽지 않으면 이해할 수 없는 내용도 많고 체계화 되지 않는 한계도 
있습니다. 

이를 보완할 수 있는 최적의 자료가 cppreference.com에서 제공하는 문서들입니다. 

구글 검색을 해도 C++ 내용들이 많이 참조되지만 직접 사이트에 들어가서 검색해도 
됩니다. 상당히 방대한 내용이 잘 정리되어 있고 표준 문서에서 불필요한 내용들도 
잘 제외되어 있습니다. 

\subsection{참고 자료들 검색}

인터넷에 매우 방대한 양의 자료들이 있습니다. std::vector로 검색해보면 매우 
많은 내용들이 나옵니다. 그 중에서 cppreference.com, 씹어먹는 C++, stackoverflow
등의 검색 내용은 거의 항상 신뢰할만한 결과를 제공합니다. 

모르면 검색하고, 알려주는 내용을 단위 테스트 내에서 확인합니다. 

\section{C++ 언어 표준은 무엇을 사용해야 하나요?}

생각보다 어려운 선택입니다. 현업에서도 그렇습니다. 현재 C++ 표준 모델이 기차 모델로 
3년마다 출발하는 기차에 탄 표준 문서들을 표준화하는 방식으로 진행됩니다. 매우 적극적으로 
표준화를 따라가는 스튜디오도 있고, 보수적으로 두 개 전의 표준을 따르는 경우도 있습니다. 
어떤 경우에는 꼭 필요한 기능이 있어 최신 표준을 따르면서 일부 기능으로 제한하여 
사용하도록 규약을 만드는 경우도 있습니다. 

이 책에서는 C++20을 기준으로 삼고 C++23의 일부를 선택하여 사용하도록 합니다. 
TODO: 구체적으로 이것이 의미하는 바는 나중에 자세히 정리하도록 합니다. 

\section{개발 환경 준비하기}

wsl(Windwos Subsystem for Linux)를 쉽게 윈도우에서도 사용할 수 있어서 clang+cmake+vscode
조합을 사용할까도 고민했습니다. 우리나라 개발환경은 대체로 Linux를 사용하는 경우에도 
데스크톱 환경은 윈도우인 경우가 많고 또 가정에서는 거의 대부분 윈도우를 사용하고 있으므로 
비주얼 스튜디오와 비주얼 C++, vcpkg 조합으로 정했습니다. 

\subsection{비주얼 스튜디오 설치}

비주얼 스튜디오 커뮤니티 배포판은 무료로 쓸 수 있고 기능 제한도 거의 없습니다. 이를 
받아서 설치하는 과정을 이미지를 따서 적고 설명할 수도 있겠으나 인터넷에 자료가 이미 
많이 있으므로 직접 진행할 수 있으리라 믿습니다. 

이와 같이 이미 알려진 방법들이 있는 경우에는 자료를 활용하도록 안내하려고 합니다. 

비주얼 스튜디오 버전은 2022를 사용합니다. 현재 C++20은 거의 대부분 구현되었고 
C++23부터는 일부만 구현되어 있습니다. 

\subsection{vcpkg 설치}

vcpkg는 마이크로소프트에서 제공하는 cmake와 비주얼 스튜디오에서 바로 사용할 수 있는 
형태로 제공되는 C++ 라이브러리 패키지들입니다. 

vcpkg 또한 인터넷 검색으로 설치 방법을 확인할 수 있습니다. 처음에 익숙하지 않은 
몇 가지 개념들이 있어 사용법 설명은 필요합니다. 

\begin{lstlisting}[language=bash, caption={vcpkg 받기}]
git clone https://github.com/Microsoft/vcpkg.git
cd vcpkg
\end{lstlisting}

\begin{lstlisting}[language=bash, caption={비주얼 스튜디오 연동하기}]
./vcpkg.exe integrate install
\end{lstlisting}

\subsection{doctest 설치}

세쌍(triplet)이라고 불리는 x64-windows-static 과 같이 지정하여 라이브러리를 x64용으로
윈도우에서 정적으로 링크할 라이브러리를 지정하여 설치할 수 있습니다. 

\begin{lstlisting}[language=bash, caption={doctest 64비트 정적 라이브러리 설치}]
    ./vcpkg.exe install doctest:x64-windows-static
\end{lstlisting}
    
게임 서버를 만들 때는 주로 위와 같이 64비트, 정적 링크를 사용하는 경우가 많습니다. 

\subsection{첫 테스트 프로젝트}

비주얼 스튜디오를 열고 C++ 콘솔 프로젝트를 생성합니다. 콘솔 프로젝트 생성 방법도 
인터넷 검색으로 찾을 수 있습니다. 

몇 가지 기본 설정은 안내할 필요가 있고, 아래와 같이 따라서 진행합니다. 










