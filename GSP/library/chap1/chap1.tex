\chapter{준비}

\section{무엇을 연습하나요?}

C++ 표준 라이브러리는 C++ 컴파일러들과 함께 배포되고 있고, 이들 라이브러리 개발자들은 
최고 수준의 C++ 프로그래머들입니다. 따라서, 제공되는 인터페이스와 설명 문서로 잘 
사용하는 연습을 통해 여러 곳에서 활용할 수 있을뿐만 아니라 C++ 언어에 대한 이해를 
높이는데 매우 효과적입니다. 

다양한 라이브러리 구성 요소들이 있습니다. 그 중에서 가장 많이 사용하는 컨테이너 자료구조와 
알고리즘을 어떻게 쓰는지 배우는 것에서 시작합니다. 필요한 라이브러리 사용법들을 추가로 
좀 더 살펴보고 연습한 후에 코드를 읽는 연습을 합니다. 

좀 익숙해지면 빅오(Big O) 기호라고 불리는 알고리즘의 성능 특성을 실제 사용 코드와 
프로파일링을 통해 확인합니다. 그 다음은 C++이 어떻게 얼마나 최적화를 잘 하는지 
어셈블리 코드를 읽으면서 확인합니다. 

% 이와 같이 진행하기는 어렵다. 
% - 간결한 키워드로 전체 항목들을 만든다. 
% - 책은 처음부터 끝까지 순차적으로 쓸 수가 없다. 
% - 프로그래밍과 마찬가지로 설계 과정을 반복하면서 개별 알고리즘을 찾고 
% - 그 다음에 서로 연결한 후에 테스트를 거치면서 열심히 개선해야 한다. 
% - 주석으로 설계하고 간결한 목록으로 내용을 만든다. 

\begin{itemize}
    \item std::array 
    \item std::vector 
    \item std::map
    \item std::unordered\_map 
    \item std::set 
\end{itemize}


\section{어떤 환경에서 연습하나요?}

% 질문과 답 형식을 최대한 유지한다. 한자어를 적게 쓴다. 순우리말과 영어 그대로 
% 사용한다. 









